\section{機械の整備}
今回作成したロボットはパーツの点数が多く,持ち運ぶ際に分解した後の組み立てに時間がかかる.そこでロボットを駆動部である2つのDモジュールと中心部のCモジュールに分解させることで,最小限の分解で持ち運びを容易に行えるように設計した.これにより組み立ての時間を短縮することが狙いである
\subsection{必要な工具}
ロボットを分解,組み立てする際に必要な工具を以下に示す.基本的に全サイズを用意しておく.
\begin{itemize}
 \item 六角レンチ
 \item レンチ
 \item 
\end{itemize}

\subsection{分解方法}
以下に分解手順を示す.本分解方法はつくばチャレンジ2016に参加した際に用いた方法と同様のものである.また,前提として上面のアクリル板に設置されたセンサ類のコードはすべて外してあり,別途に確保してある状態とする.

\subsubsection{ポールとアクリル板の取り外し}
ロボット上面のアクリル板はM3の六角穴付ボルトを使用している.ポールはM6の六角穴付ボルトとM3のボルトを使用している.まずアクリル板に取り付けてあるM3を全て取り外し,ポールとアクリル板を一緒に外す.このとき,ポールそのものを持ち,運び上げる行為は禁止である.ポールが外れアクリル板が落下し破損する恐れがある.取り外したのち腹面のアクリル板も同様に取り外す.
\subsubsection{DモジュールとCモジュールの分解}
この分解はDモジュールに接続されているコード類を引きちぎらないように注意を払って分解する.
ロボットの中心に保持されている部分がCモジュールである.CモジュールはM8の六角穴付ボルト4本で2つのDモジュールと接続されている.このM8ボルトのナットを全て取り外し,Dモジュールを横に引き出す.このときDモジュールに接続されたコード類を取り外した後,Cモジュールを取り出す.このときボルトはDモジュールと一体化しているが問題はなく,またカラーを無くさないようにする.くれぐれもコード類を引きちぎらないように注意する.

\subsubsection{ポール分解}
ポールは支持パーツとカメラも接続されているため,それも分解する.ポールはM6のボルトを使用している.底面に取り付けられている木の板を取り外し,支持パーツをスライドさせながら取り出す.この時支持パーツの中に1本だけ長いものがあるため,気をつけて整理する.カメラを保持するパーツはコードとボルトを取り外し舗装する.

\subsubsection{センサの取り外し}
上面アクリル板はモーションセンサと測域センサを取り付けてあるため,これも外し個包装を施す.

\subsubsection{箱つめ}
分解したD,Cモジュールを箱に詰める.詰める際にはCモジュールに接続されたセンサ関連は全て取り外す.

