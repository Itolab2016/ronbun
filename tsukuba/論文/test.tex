\section{今後の課題}
\subsection{自己位置推定}
現在作成できたマップには旋回時に大きく歪むという欠点がある.そのため旋回時には他のセンサーからの情報で外部パラメータ行列に補正をかける必要がある.そこで,今回はタイヤの滑りが原因で使用できなかったがエンコーダなどを使用する方法が考えられる.また,マップ作成時に使用した画像から一定間隔でテンプレート画像を抽出しそれと現在の画像をテンプレートマッチングできれば,対応したテンプレート画像の位置から自己位置を推定できる.この場合,現在の画像とテンプレート画像は鳥瞰図のように同じ視点からの画像になるように透視変換する必要があるが,前後のテンプレート画像に比較対象を絞ることができるため精度が高くなると考えられる.ただし,スタート時には参照できる前後のテンプレート画像を自力で用意できない問題がある他,テンプレートマッチング自体の計算量が多く,処理時間がかかってしまうことが考えられる.
