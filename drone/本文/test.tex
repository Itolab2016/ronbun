\documentclass[12pt,oneside]{sotsuken_paper}

% タイトル
\title{安全な教育用マルチコプターの開発}
\author{北山天斗・剱崎健太郎}

\begin{document}
% 行間
\setlength{\baselineskip}{9truemm}

%文字間
\kanjiskip=.53zw plus 3pt minus 3pt
\xkanjiskip=.53zw plus 3pt minus 3pt

% 目次
\tableofcontents
%\newpage

% 本文

\chapter{はじめに}
\section{研究背景}
\section{研究目的}

\chapter{ハードウェア}
\section{目標}
\section{仕様}
\subsection{機体}
\subsection{球体}
\section{経緯}
\subsection{機体}
\subsubsection{試作1号機}
\subsubsection{試作2号機}
\subsubsection{試作3号機}
\subsubsection{試作3号機改}
\subsection{球体}
\subsubsection{試作1号機}
\subsubsection{試作2号機}
\subsubsection{試作3号機}
\section{結果}
\section{今後の課題}

\chapter{ソフトウェア}
\section{目標}
\section{概要}
\section{開発環境}
\section{PID制御}
\section{飛行試験について}
\section{運動方程式}
\section{特性確認}
\subsection{物理パラメータ}
\subsection{推力特性}
\subsection{モーメント特性}
\section{校正}
\subsection{加速度センサ}
\subsection{地磁気センサ}
\section{最適レギュレータ}
\section{結果}
\section{今後の課題}

\chapter{おわりに}

\chapter*{謝辞}
\addcontentsline{toc}{chapter}{謝辞}
本論文作成にあたりテーマの決定,研究の考え方,方法のまとめ方など全てにおいて長期にわたって厳しくも熱意のあるご指導,ご鞭撻していただいた,伊藤恒平教授に厚く御礼申し上げます.


特に分析においても論文の書き方においても論文を何度も読んでいただき,指導していただいた伊藤恒平教授に大変ご苦労をかけてしまいましたことにも心よりお詫び申し上げたいです.


同級生のメンバーには論文の作成,修正にご協力いただき心より感謝しております.
その他、助けていただいた多くの皆様に心から感謝しております.ありがとうございました.

\chapter*{付録}

\end{document}

